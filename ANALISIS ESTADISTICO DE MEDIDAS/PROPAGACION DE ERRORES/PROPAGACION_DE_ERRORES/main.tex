\documentclass[%
 reprint,
%superscriptaddress,
groupedaddress,
unsortedaddress,
%runinaddress,
%frontmatterverbose, 
%preprint,
%preprintnumbers,
%nofootinbib,
%nobibnotes,
%bibnotes,
 amsmath,amssymb,
 aps,
%pra,
%prb,
%rmp,
%prstab,
%prstper
%floatfix,
superscriptaddress
]{revtex4-2}
\include{Formato}
\usepackage[spanish]{babel}
\usepackage{graphicx}
\usepackage{float}
\usepackage{diagbox}
\usepackage[backend=bibtex]{biblatex}
\bibliography{bibliografia}
\pagestyle{plain}
\begin{document}
\renewcommand{\tablename}{Tabla}

\preprint{APS/123-QED}

\title{PROPAGACION DE ERRORES}

\author{Merino Vidal, Mateo}
\affiliation{Departamento de Informática y Sistemas }
\email{202301308@est.umss.edu}

\date{\today}

\begin{abstract}
\hfill \break
Se realizo el presente trabajo mediante el análisis de varias series de medidas directas tomadas de las dimensiones de diversas figuras tridimensionales como la esfera, el cilindro hueco, el cilindro y la moneda.
Las cuales se utilizaron con diversos cálculos estadísticos, permitiendo obtener diversas medidas indirectas como el volumen, el peso y la densidad.\\
Asimismo también se logro determinar el error existente en la serie de medidas indirectas, a través del método de propagación de errores.
En otras palabras, se determino la propagación o efecto que producen los errores de las mediciones directas al error de la función.

Como en el caso del test3 del cilindro en la Tabla.\ref{Tabla31} con un mínimo error de la función del volumen de: $0.00014$. 

\begin{description}
\item[Palabras clave]Media, Medidas Directas, Medidas Indirectas, Propagación de Errores, Error de la Media.
\end{description}
\end{abstract}

\maketitle

\section{Introducción}
\label{sec:introduccion}
\hfill \break
Los datos presentados en este informe, son una extensión del trabajo de medidas indirectas, realizado por Mateo Merino Vidal \cite{202301308Prt0230032023}, presentado durante el curso de laboratorio de física general del departamento de Física, cabe mencionar que se trabajo con análisis básicos para la presentación final de datos con diferentes tipos de dispersión.
\hfill \break
\hfill \break
En el ámbito científico es común el realizar mediciones repetidas de una o más variables, cada una con sus incertidumbres individuales. Estas incertidumbres son de tipo instrumental que se puede conocer dadas las características del instrumento de medición, o de tipo aleatorio que proviene de numerosos factores que en la mayoría de las ocasiones son difíciles de controlar.
\hfill \break
\hfill \break
Las magnitudes de origen experimental pueden ser de dos tipos.
\hfill \break
\hfill \break
Por una parte, están las magnitudes directas, que son las que se obtienen mediante la comparación directa con el instrumento de medición. 
\hfill \break
\hfill \break
Por otro lado, están las magnitudes indirectas, que dependen del cálculo de una función de magnitudes directas, como cuando se calcula la velocidad a partir de mediciones de tiempo y distancia.
\hfill \break
\hfill \break
En la mayoría de las mediciones físicas se busca determinar cantidades que se obtienen mediante el calculo a partir de una o varias cantidades medidas directamente dando así origen a un resultado indirecto.
La estimación del error del resultado final a partir de los errores de las cantidades medidas directamente se conoce como propagación de errores, tal y como expresan Pedro M. Diaz R y Edwin A. Torrellas V \cite{Barquisimeto}.

\begin{itemize}
\item \textbf{Propagación de errores:} Conjunto de reglas que permiten asignar un error a z, conocidas las incertidumbres de x e y.
\hfill \break
\hfill \break
- Permiten asignar un error al resultado final.
\hfill \break
\hfill \break
- Indica la importancia relativa de las diferentes medidas directas.
\hfill \break
\hfill \break
- Planificación del experimento. 
\end{itemize}
\hfill \break
La propagación de errores, o de incertidumbres, se asocia con la obtención de nuevos resultados a partir de un cálculo con las cantidades medidas y su respectivo error.
Uno de los métodos utilizados para calcular errores en las mediciones es el metodo de las derivadas parciales, dependiente de la cantidad de variables cualitativas y la operación entre estas, tal y como expresa.

\begin{itemize}
\item \textbf{Medidas Cualitativas:} Describe una característica como una cualidad o atributo.\\
Es una cualidad que el sujeto posee o no posee.
\end{itemize}
\hfill \break
Segun la posicion en la relacion que une una a dos o mas variables, estas se clasifican en dependientes e independientes, tal y como explica Eleonora Espinoza \cite{Ale}.
\begin{itemize}
\item \textbf{Dependientes:} Son las que designan las variables a explicar, los efectos o resultados respecto a las cuales hay que buscar el motivo o razón, dependen de las variables independientes.
\item \textbf{Independientes:} Son las variables explicativas cuya asociación o influencia en la variable dependiente se pretende descubrir.
\end{itemize}

\section{Objetivo y planteamiento del problema}
\hfill \break
El presente trabajo busca determinar la estimación del error de la funcion en la serie de medidas indirectas del volumen y la densidad de las diversas figuras tridimensionales como la esfera, el cilindro hueco, el cilindro y la moneda.
\hfill \break
\hfill \break
Utilizando los valores de la media y su error previamente calculadas en el trabajo de medidas indirectas \cite{}, mediante el análisis de datos en torno a bases estadísticas y el planteamiento de ecuaciones. 
\hfill \break
\hfill \break
Tomando en cuenta las reglas del metodo de propagacion de erorres, mediante las derivadas parciales, dependiendo de la cantidad de variables independientes que tenga la medida indirecta de cada figura tridimensional en el volumen y la densidad, hallando primeramente las contribuciones individuales de cada una al error de la funcion.
\hfill \break
\hfill \break
Posteriormente se llega analizar los resultados obtenidos en cuanto al error de la funcion en cada uno de los tests de las diversas figuras tridimensionales mediante diversos histogramas, con la finalidad de determinar la gravedad del error en los tests.
\hfill \break
 \hfill \break
 
\section{Método}
\label{sec:met}
\hfill \break
\textbf{Formulas de Volumen y Densidad}
\begin{gather}
V_E=\dfrac{\pi \times D^3}{6} 
\label{ec:1} \\ \notag \\ 
V_{Ci}=\dfrac{\pi \times D^2 \times H}{4} 
\label{ec:2}\\ \notag \\ 
V_{Ci-Hu}=\dfrac{\pi \times (D^2-d^2)\times H}{4}
\label{ec:3}\\ \notag \\ 
V_{Mon}=\dfrac{\pi \times D^2 \times E}{4}
\label{ec:4}\\ \notag \\ 
D=\dfrac{M}{V}
\label{ec:5}
\end{gather} 
\hfill \break
\textbf{Formula de la Media}
\begin{gather}
X_{Rep}=\dfrac{1}{N}\times \sum_{i=1}^N X_i
\label{ec:7}
\end{gather}
\hfill \break
\textbf{Formula de la Desviación Individual}
\begin{gather}
d_i=X_i-X_{rep}
\label{ec:8}
\end{gather}
\hfill \break
\textbf{Formulas de la Desviación Típica para: Gran cantidad de datos, Pequeña cantidad de datos}
\begin{gather}
\sigma=\sqrt{\dfrac{1}{N}\times\sum{d_i} ^2}
\label{ec:9}\\ \notag \\
\sigma_{n-1} =\sqrt{\dfrac{1}{N-1}\times\sum{d_i} ^2}
\label{ec:10}
\end{gather}
\hfill \break
\textbf{Formulas del error de la media aritmética para: Gran cantidad de datos, Pequeña cantidad de datos}
\begin{gather}
\sigma_x=\dfrac{\sigma}{\sqrt{N}} 
\label{ec:11}\\ \notag \\
\sigma_x=\dfrac{\sigma_{n-1}}{\sqrt{N}}
\label{ec:12}
\end{gather}
\hfill \break
\textbf{Formulas para determinar: La contribución de las variables independientes al error de la función, La estimación del error de la función}
\begin{gather}
\Delta x=|\dfrac{\partial f}{\partial x}|e_x_\prime
\label{ec:13}\\ \notag \\
\Delta y=|\dfrac{\partial f}{\partial y}|e_y_\prime
\label{ec:14}\\ \notag \\
\Delta z=|\dfrac{\partial f}{\partial z}|e_z_\prime
\label{ec:15}\\ \notag \\
E_f=\sqrt{\Delta x^2 +\Delta y^2 +\Delta z^2 + ....}
\label{ec:16}
\end{gather}

\subsection{MEDICIONES DE LA ESFERA}
\hfill \break
Obteniendo datos de la media del diámetro y la masa de la esfera, previamente calculados en el trabajo de medidas indirectas realizada por Mateo Merino Vidal\cite{202301308Prt0230032023}, a través de la Ec.~\eqref{ec:7}.

\begin{table}[H]
\centering
\begin{tabular}{|c|c|c|}
\hline
Test & Diámetro [cm] & Masa [gr]\\
\hline
Esfera1 & 1.44 & 15.11\\
\hline
Esfera2 & 3.31 & 34.49\\
\hline
Esfera3 & 3.39 & 48.10\\
\hline
Esfera4 & 3.87 & 64.89\\
\hline
Esfera5 & 4.64 & 74.38\\
\hline
Esfera6 & 5.55 & 75.26\\
\hline
Esfera7 & 7.33 & 102.084\\
\hline
Esfera8 & 7.73 & 113.40\\
\hline
Esfera9 & 8.32 & 113.79\\
\hline
Esfera10 & 9.49 & 121.00\\
\hline
\end{tabular}
\caption{\centering Valor de tendencia central de los datos obtenidos del diámetro y la masa de la esfera en cada test.}
\label{tabla1}
\end{table}
\hfill \break
Obteniendo datos de la media del volumen de la esfera previamente calculados en el trabajo de medidas indirectas realizada por Mateo Merino Vidal\cite{202301308Prt0230032023}, a través de la Ec.~\eqref{ec:1}, para hallar los volúmenes individuales en base a las medidas directas obtenidas de la medición de las dimensiones de la esfera y la Ec.~\eqref{ec:7}, para hallar la media de esos volúmenes en cada test de la esfera.

\begin{table}[H]
\centering
\begin{tabular}{|c|c|}
\hline
Test & Media del Volumen [$cm^3$]\\
\hline
Esfera1 & 1.56 \\
\hline
Esfera2 & 18.94\\
\hline
Esfera3 & 20.48\\
\hline
Esfera4 & 30.27\\
\hline
Esfera5 & 52.38\\
\hline
Esfera6 & 89.40\\
\hline
Esfera7 & 206.28\\
\hline
Esfera8 & 241.96\\
\hline
Esfera9 & 301.63\\
\hline
Esfera10 & 447.06\\
\hline
\end{tabular}
\caption{ \centering Promedio del volumen en cada test}
\label{Tabla2}
\end{table}

\begin{figure}[H]
\centering
\includegraphics[scale=0.55]{Imagen1.png}
\caption{\centering Media del volumen de la esfera en cada test}
\label{Figura1}
\end{figure}
\hfill \break
Obteniendo datos de la media de la densidad de la esfera previamente calculados en el trabajo de medidas indirectas realizada por Mateo Merino Vidal\cite{202301308Prt0230032023}, a través de la Ec.~\eqref{ec:5}, para hallar las densidades individuales en base a los datos obtenidos de masa, volumen y la Ec.~\eqref{ec:7}, para hallar la media de esas densidades en cada test de la esfera.

\begin{table}[H]
\centering
\begin{tabular}{|c|c|}
\hline
Test & Media de la Densidad [$gr/cm^3$]\\
\hline
Esfera1 & 9.70\\
\hline
Esfera2 & 1.82\\
\hline
Esfera3 & 2.34\\
\hline
Esfera4 & 2.14\\
\hline
Esfera5 & 1.42\\
\hline
Esfera6 & 0.84\\
\hline
Esfera7 & 0.49 \\
\hline
Esfera8 & 0.47 \\
\hline
Esfera9 & 0.38 \\
\hline
Esfera10 & 0.27\\
\hline
\end{tabular}
\caption{ \centering Promedio de la densidad de cada test}
\label{Tabla3}
\end{table}

\begin{figure}[H]
\centering
\includegraphics[scale=0.55]{Imagen2.png}
\caption{\centering Media de la densidad de la esfera en cada test}
\label{Figura2}
\end{figure}
\hfill \break
Aplicando la ecuación ~\eqref{ec:8}, utilizando la media del diámetro y la masa de la esfera obtenidos anteriormente, con la finalidad de encontrar sus desviaciones individuales en cada uno de los tests.
\hfill \break
\hfill \break
Aplicando la Ec. ~\eqref{ec:9} a las desviaciones individuales del diámetro y la masa obtenidos en cada test de la esfera, con la finalidad de determinar la desviación típica de cada uno en los diversos tests.
\hfill \break
\hfill \break
Aplicando la Ec.~\eqref{ec:11} a las desviaciones típicas del diámetro y la mas obtenidos anteriormente en cada test, con la finalidad de calcular el error de la media en cada uno de los tests de la esfera.

\hfill \break
\textbf{Diámetro} 

\begin{table}[H]
\centering
\begin{tabular}{|l|c|c|c|}
\hline
\backslashbox{Test}{Cálculos} & Media [cm] & Des.Estándar & Error.Media\\
\hline
Esfera1 & 1.44  & 0.00044	& 2.2e-05	\\
\hline
Esfera2 & 3.31  & 0.00054	& 2.7e-05	\\
\hline
Esfera3 & 3.39  & 0.00011	& 	5.6e-06\\
\hline
Esfera4 & 3.87  & 0.0011	& 	5.6e-05\\
\hline
Esfera5 & 4.64  & 0.00083	& 	4.2e-05\\
\hline
Esfera6 & 5.55  & 0.00037	&  1.9e-05	\\
\hline
Esfera7 & 7.33  & 0.00085	&  4.3e-05	\\
\hline
Esfera8 &  7.73 & 0.00098	& 	4.9e-05\\
\hline
Esfera9 &  8.32 & 0.00071	& 3.6e-05	\\
\hline
Esfera10 &  9.49 & 0.00064	&  3.2e-05	\\
\hline
\end{tabular}
\caption{\centering Punto de tendencia central, desviaciones y error de la media en el diámetro de la esfera en cada test}
\label{Tabla4}
\end{table}
\hfill \break
\textbf{Masa} 

\begin{table}[H]
\centering
\begin{tabular}{|l|c|c|c|}
\hline
\backslashbox{Test}{Cálculos} & Media [gr] & Des.Estándar & Error.Media\\
\hline
Esfera1 & 15.11  & 0.026	& 0.0013	\\
\hline
Esfera2 & 34.50  & 0.0048	& 0.00024	\\
\hline
Esfera3 & 48.10  & 0.020	& 0.0010	\\
\hline
Esfera4 & 64.89  & 0.0034	& 0.00017	\\
\hline
Esfera5 & 74.38  & 0.0083	& 0.00041	\\
\hline
Esfera6 &  75.26 & 0.013 	& 0.00065	\\
\hline
Esfera7 &  102.08 & 0.011	& 0.00056	\\
\hline
Esfera8 & 113.40  & 0.011	& 0.00054	\\
\hline
Esfera9 & 113.79  & 0.019	& 0.00096	\\
\hline
Esfera10 &  121.00 & 0.011	& 0.00057	\\
\hline
\end{tabular}
\caption{\centering Punto de tendencia central, desviaciones y error de la media en la masa de la esfera en cada test}
\label{Tabla5}
\end{table}
\hfill \break
Obteniendo datos del error de la media del volumen, previamente calculados en el trabajo de medidas indirectas realizada por Mateo Merino Vidal\cite{202301308Prt0230032023}, a través de la Ec.~\eqref{ec:11}, utilizando la desviación típica calculada mediante la Ec.~\eqref{ec:9} en base a las desviaciones individuales de los volúmenes, determinadas mediante la Ec.~\eqref{ec:8}.

\hfill \break
\textbf{Volumen} \\

\begin{table}[H]
\centering
\begin{tabular}{|l|c|}
\hline
\backslashbox{Test}{Cálculos}& Error de la Media\\
\hline
Esfera1 & 7.0e-05	\\
\hline
Esfera2 & 0.00046	\\
\hline
Esfera3 & 0.00010	\\
\hline
Esfera4 & 0.0013	\\
\hline
Esfera5 & 0.0014	\\
\hline
Esfera6 & 0.00091	\\
\hline
Esfera7 & 0.0036	\\
\hline
Esfera8 & 0.0046	\\
\hline
Esfera9 & 0.0039	\\
\hline
Esfera10 & 0.0045	\\
\hline
\end{tabular}
\caption{\centering Punto de tendencia central, desviaciones y error de la media en el volumen de la esfera en cada test}
\label{Tabla6}
\end{table}
\hfill \break
Aplicando la Ec.~\eqref{ec:13} a la ecuación del volumen Ec.~\eqref{ec:1}, con la finalidad de calcular la contribución del diámetro al error de la función del volumen, utilizando las medias y los errores de la media del diámetro obtenidos en cada test de la esfera.

\begin{table}[H]
\centering
\begin{tabular}{|l|c|}
\hline
\backslashbox{Test}{Cálculos}& Contribución del Diametro\\
\hline
Esfera1 & 	7.1e-05\\
\hline
Esfera2 & 	0.00046\\
\hline
Esfera3 & 	0.00010\\
\hline
Esfera4 & 	0.0013\\
\hline
Esfera5 & 	0.0014\\
\hline
Esfera6 & 	0.00091\\
\hline
Esfera7 & 	0.0036\\
\hline
Esfera8 & 	0.0046\\
\hline
Esfera9 & 	0.0039\\
\hline
Esfera10 &  0.0045\\
\hline
\end{tabular}
\caption{\centering Contribución del diámetro al error de la función del volumen en cada test de la esfera}
\label{Tabla7}
\end{table}
\hfill \break
Aplicando la Ec.~\eqref{ec:16}, con la finalidad de calcular el error de la función del volumen, utilizando las contribuciones del diámetro calculadas en cada test de la esfera.

\begin{table}[H]
\centering
\begin{tabular}{|l|c|}
\hline
\backslashbox{Test}{Cálculos}& Error en la función del Volumen\\
\hline
Esfera1 & 	7.1e-05\\
\hline
Esfera2 & 	0.00046\\
\hline
Esfera3 & 	0.00010\\
\hline
Esfera4 & 	0.0013\\
\hline
Esfera5 & 	0.0014\\
\hline
Esfera6 & 	0.00091\\
\hline
Esfera7 & 	0.0036\\
\hline
Esfera8 & 	0.0046\\
\hline
Esfera9 & 	0.0039\\
\hline
Esfera10 &  0.0045\\
\hline
\end{tabular}
\caption{\centering Error en la función del volumen en cada test de la esfera}
\label{Tabla8}
\end{table}
\hfill \break
Aplicando la Ec.~\eqref{ec:13} y la Ec.~\eqref{ec:14} a la ecuación de la densidad Ec.~\eqref{ec:5}, con la finalidad de calcular la contribución de la masa y del volumen al error de la función de la densidad, utilizando las medias y los errores de la media de la masa y el volumen obtenidos en cada test de la esfera.

\begin{table}[H]
\centering
\begin{tabular}{|l|c|c|}
\hline
\backslashbox{Test}{Cálculos}& Contribución.Masa  & Contribución.Volumen\\
\hline
Esfera1 & 0.00084	 & 0.00044\\
\hline
Esfera2 & 1.3e-05	 &4.4e-05\\
\hline
Esfera3 & 4.9e-05  & 1.2e-05\\
\hline
Esfera4 &  5.6e-06	 & 9.2e-05\\
\hline
Esfera5 &  7.9e-06	 & 3.8e-05\\
\hline
Esfera6 & 7.3e-06	 &  8.5e-06\\
\hline
Esfera7 & 2.7e-06	 & 8.6e-06\\
\hline
Esfera8 & 2.2e-06	 & 8.9e-06\\
\hline
Esfera9 & 3.2e-06	 & 4.8e-06\\
\hline
Esfera10 & 1.3e-06	 & 2.7e-06\\
\hline
\end{tabular}
\caption{\centering Contribución de la masa y el volumen al error de la función de la densidad en cada test de la esfera}
\label{Tabla9}
\end{table}
\hfill \break
Aplicando la Ec.~\eqref{ec:16}, con la finalidad de calcular el error de la función de la densidad, utilizando las contribuciones de la masa y el volumen calculadas en cada test de la esfera.

\begin{table}[H]
\centering
\begin{tabular}{|l|c|}
\hline
\backslashbox{Test}{Cálculos}& Error en la función de la Densidad\\
\hline
Esfera1 & 	0.00094\\
\hline
Esfera2 & 	 4.6e-05\\
\hline
Esfera3 & 	5.0e-05\\
\hline
Esfera4 & 	9.3e-05\\
\hline
Esfera5 & 	3.9e-05\\
\hline
Esfera6 & 	1.1e-05\\
\hline
Esfera7 & 	9.1e-06\\
\hline
Esfera8 & 	9.2e-06\\
\hline
Esfera9 & 	5.8e-06\\
\hline
Esfera10 & 	3.0e-06\\
\hline
\end{tabular}
\caption{\centering Error en la función de la Densidad en cada test de la esfera}
\label{Tabla10}
\end{table}

\subsection{MEDICIONES DEL CILINDRO HUECO}
\hfill \break
Obteniendo datos de la media de la altura, radio externo, radio interno y la masa del cilindro hueco, previamente calculados en el trabajo de medidas indirectas realizada por Mateo Merino Vidal\cite{202301308Prt0230032023}, a través de la Ec.~\eqref{ec:7}.

\begin{table}[H]
\centering
\begin{tabular}{|c|c|c|c|c|}
\hline
Test & Altura [cm] & Radio.In [cm] & Radio.Ex [cm] & Masa [gr]\\
\hline
Cilin.H1 & 3.66 & 1.08 & 2.25 & 45.57\\
\hline
Cilin.H2 & 4.25 & 1.25 & 2.67 & 55.23\\
\hline
Cilin.H3 & 5.20 & 1.38 & 3.05 & 58.21\\
\hline
Cilin.H4 & 5.87 & 1.45 & 3.37 & 81.18\\
\hline
Cilin.H5 & 7.34 & 1.84 & 3.42 & 89.54\\
\hline
Cilin.H6 & 8.15 &  2.29 & 3.62  & 105.74\\
\hline
Cilin.H7 & 8.60 & 2.36 & 3.76 & 108.85\\
\hline
Cilin.H8 & 8.73 &  2.48 & 4.45 & 128.45\\
\hline
Cilin.H9 & 9.04 & 2.74 & 5.35 & 128.96\\
\hline
Cilin.H10 & 9.90 & 4.08 & 5.90 & 139.76\\
\hline
\end{tabular}
\caption{\centering Valor de tendencia central de los datos obtenidos de la altura, el radio exterior, el radio interior y la masa del cilindro hueco en cada test.}
\label{tabla11}
\end{table}
\hfill \break
Obteniendo datos de la media del volumen del cilindro hueco, previamente calculados en el trabajo de medidas indirectas realizada por Mateo Merino Vidal\cite{202301308Prt0230032023}, a través de la Ec.~\eqref{ec:3}, para hallar los volúmenes individuales en base a las medidas directas obtenidas de la medición de las dimensiones del cilindro hueco y la Ec.~\eqref{ec:7}, para hallar la media de esos volúmenes en cada test del cilindro hueco.

\begin{table}[H]
\centering
\begin{tabular}{|c|c|}
\hline
Test & Media del Volumen [$cm^3$]\\
\hline
Cilin.H1 & 45.04\\
\hline
Cilin.H2 & 74.26\\
\hline
Cilin.H3 & 120.89\\
\hline
Cilin.H4 & 170.39\\
\hline
Cilin.H5 & 192.41\\
\hline
Cilin.H6 & 200.28\\
\hline
Cilin.H7 & 231.48\\
\hline
Cilin.H8 & 375.38\\
\hline
Cilin.H9 & 600.11\\
\hline
Cilin.H10 & 564.75\\
\hline
\end{tabular}
\caption{ \centering Promedio del volumen de cada test}
\label{Tabla12}
\end{table}

\begin{figure}[H]
\centering
\includegraphics[scale=0.55]{Imagen6.png}
\caption{\centering Media del volumen del cilindro hueco en cada test}
\label{Figura3}
\end{figure}
\hfill \break
Obteniendo datos de la media de la densidad del cilindro hueco previamente calculados en el trabajo de medidas indirectas realizada por Mateo Merino Vidal\cite{202301308Prt0230032023}, a través de la Ec.~\eqref{ec:5}, para hallar las densidades individuales en base a los datos obtenidos de masa, volumen y la Ec.~\eqref{ec:7}, para hallar la media de esas densidades en cada test del cilindro hueco.

\begin{table}[H]
\centering
\begin{tabular}{|c|c|}
\hline
Test & Media de la Densidad [$gr/cm^3$]\\
\hline
Cilin.H1 & 1.00\\
\hline
Cilin.H2 & 0.74 \\
\hline
Cilin.H3 & 0.48\\
\hline
Cilin.H4 & 0.48 \\
\hline
Cilin.H5 & 0.47 \\
\hline
Cilin.H6 & 0.53\\
\hline
Cilin.H7 & 0.47\\
\hline
Cilin.H8 & 0.34\\
\hline
Cilin.H9 & 0.21\\
\hline
Cilin.H10 & 0.25\\
\hline
\end{tabular}
\caption{ \centering Promedio de la densidad de cada test}
\label{Tabla13}
\end{table}

\begin{figure}[H]
\centering
\includegraphics[scale=0.55]{Imagen7.png}
\caption{\centering Media de la densidad del cilindro hueco en cada test}
\label{Figura4}
\end{figure}
\hfill \break
Aplicando la ecuación ~\eqref{ec:8}, utilizando la media de la altura, el radio interno, el radio externo y la masa del cilindro hueco obtenidos anteriormente, con la finalidad de encontrar sus desviaciones individuales en cada uno de los tests.
\hfill \break
\hfill \break
Aplicando la Ec. ~\eqref{ec:9} a las desviaciones individuales de la altura, el radio interno, el radio externo y la masa obtenidos en cada test del cilindro hueco, con la finalidad de determinar la desviacion tipica de cada uno en los diversos tests.
\hfill \break
\hfill \break
Aplicando la Ec.~\eqref{ec:11} a las desviaciones tipicas de la altura, el radio interno, el radio externo y la masa obtenidos anteriormente en cada test, con la finalidad de calcular el error de la media en cada uno de los tests del cilindro hueco.

\hfill \break
\textbf{Radio Interno} 

\begin{table}[H]
\centering
\begin{tabular}{|l|c|c|c|}
\hline
\backslashbox{Test}{Cálculos} & Media [cm] & Des.Estándar & Error.Media\\
\hline
Cilin.H1 & 1.1  & 7.6e-05	& 3.8e-06	\\
\hline
Cilin.H2 & 1.25  & 0.00010	& 	5.0e-06\\
\hline
Cilin.H3 & 1.38  & 5.1e-05	& 2.5e-06	\\
\hline
Cilin.H4 &  1.45 & 0.00010	& 5.0e-06	\\
\hline
Cilin.H5 & 1.84  & 0.00055	& 2.8e-05	\\
\hline
Cilin.H6 & 2.29  & 0.00019	& 9.6e-06	\\
\hline
Cilin.H7 & 2.36  & 9.2e-05	& 4.6e-06	\\
\hline
Cilin.H8 & 2.48 & 0.00030 	& 1.5e-05	\\
\hline
Cilin.H9 & 2.74  & 0.00075	& 3.7e-05	\\
\hline
Cilin.H10 &  4.08 & 0.00024	& 1.2e-05	\\
\hline
\end{tabular}
\caption{\centering Punto de tendencia central, desviaciones y error de la media del radio interno del cilindro hueco en cada test}
\label{Tabla14}
\end{table}
\hfill \break
\textbf{Radio Externo} 

\begin{table}[H]
\centering
\begin{tabular}{|l|c|c|c|}
\hline
\backslashbox{Test}{Cálculos} & Media [cm] & Des.Estándar & Error.Media\\
\hline
Cilin.H1 &  2.25  & 0.00018	& 8.9e-06	\\
\hline
Cilin.H2 & 2.67  & 0.00033	&  1.7e-05	\\
\hline
Cilin.H3 & 3.05  & 0.00049	& 2.5e-05	\\
\hline
Cilin.H4 & 3.37  & 7.0e-06	& 3.5e-07	\\
\hline
Cilin.H5 &  3.42  &  6.1e-05	& 3.1e-06	\\
\hline
Cilin.H6 & 3.6  & 0.00053	& 2.7e-05	\\
\hline
Cilin.H7 & 3.8  & 0.00062	& 3.1e-05	\\
\hline
Cilin.H8 & 4.4  & 0.00025	&  1.3e-05	\\
\hline
Cilin.H9 & 5.4  & 0.00032 	& 1.6e-05	\\
\hline
Cilin.H10 & 5.9  & 0.00038	&  1.9e-05	\\
\hline
\end{tabular}
\caption{\centering Punto de tendencia central, desviaciones y error de la media del radio externo del cilindro hueco en cada test}
\label{Tabla15}
\end{table}
\hfill \break
\textbf{Altura} 

\begin{table}[H]
\centering
\begin{tabular}{|l|c|c|c|}
\hline
\backslashbox{Test}{Cálculos} & Media [cm] & Des.Estándar & Error.Media\\
\hline
Cilin.H1 &  3.66 & 0.00096	& 4.79-05	\\
\hline
Cilin.H2 & 4.25  & 0.0020	& 0.00010	\\
\hline
Cilin.H3 & 5.20  & 0.00010	& 5.1e-06	\\
\hline
Cilin.H4 &  5.87  & 0.00039	& 1.9e-05	\\
\hline
Cilin.H5 &  7.34 &  0.00046	& 2.3e-05	\\
\hline
Cilin.H6 &  8.15  & 9.4e-05	&  4.7e-06	\\
\hline
Cilin.H7 & 8.60  &  0.00029	& 1.5e-05	\\
\hline
Cilin.H8 &  8.73 &  0.00060	& 3.0e-05	\\
\hline
Cilin.H9 &  9.04  & 0.0010	& 5.1e-05	\\
\hline
Cilin.H10 & 9.90  &  0.00050	&  2.5-05	\\
\hline
\end{tabular}
\caption{\centering Punto de tendencia central, desviaciones y error de la media de la altura del cilindro hueco en cada test}
\label{Tabla16}
\end{table}
\hfill \break
\textbf{Masa} 

\begin{table}[H]
\centering
\begin{tabular}{|l|c|c|c|}
\hline
\backslashbox{Test}{Cálculos} & Media [gr] & Des.Estándar & Error.Media\\
\hline
Cilin.H1 & 45.57  & 0.010	& 0.00052	\\
\hline
Cilin.H2 & 55.23  & 0.017	&  0.00087	\\
\hline
Cilin.H3 &  58.21 & 0.0077	& 0.00038	\\
\hline
Cilin.H4 &  81.18 & 0.0036	& 	 0.00018\\
\hline
Cilin.H5 &  89.54  & 0.0030	& 0.00015	\\
\hline
Cilin.H6 &  105.74  & 0.010	&  0.00051	\\
\hline
Cilin.H7 &  108.85  & 0.015	& 0.00075	\\
\hline
Cilin.H8 &  128.45  & 0.014	& 0.00069	\\
\hline
Cilin.H9 &  128.96 & 0.00034	& 1.7e-05	\\
\hline
Cilin.H10 &  139.75 & 0.011 	& 	 0.00057\\
\hline
\end{tabular}
\caption{\centering Punto de tendencia central, desviaciones y error de la media en la masa de la cilindro hueco en cada test}
\label{Tabla17}
\end{table}
\hfill \break
Obteniendo datos del error de la media del volumen, previamente calculados en el trabajo de medidas indirectas realizada por Mateo Merino Vidal\cite{202301308Prt0230032023}, a traves de la Ec.~\eqref{ec:11}, utilizando la desviacion tipica calculada mediante la Ec.~\eqref{ec:9} en base a las desviaciones individuales de los volumenes, determinadas mediante la Ec.~\eqref{ec:8}.

\hfill \break
\textbf{Volumen} 

\begin{table}[H]
\centering
\begin{tabular}{|l|c|}
\hline
\backslashbox{Test}{Cálculos} & Error.Media\\
\hline
Cilin.H1 & 3.4e-05\\
\hline
Cilin.H2 & 0.00041	\\
\hline
Cilin.H3 &  0.0025	\\
\hline
Cilin.H4 & 0.00087\\
\hline
Cilin.H5 &  0.0013	\\
\hline
Cilin.H6 &  0.0062	\\
\hline
Cilin.H7 &  0.0061	\\
\hline
Cilin.H8 &  0.0064	\\
\hline
Cilin.H9 &  0.0024	\\
\hline
Cilin.H10 &  0.0025	\\
\hline
\end{tabular}
\caption{\centering Punto de tendencia central, desviaciones y error de la media en el volumen del cilindro hueco en cada test}
\label{Tabla18}
\end{table}
\hfill \break
Aplicando la Ec.~\eqref{ec:13},Ec.~\eqref{ec:14},Ec.~\eqref{ec:15} a la ecuación del volumen Ec.~\eqref{ec:3}, con la finalidad de calcular la contribución del radio interno, el radio externo y la altura al error de la función del volumen, utilizando las medias y los errores de la media de los mismos obtenidos en cada test del cilindro hueco.

\begin{table}[H]
\centering
\begin{tabular}{|l|c|c|c|}
\hline
\backslashbox{Test}{Cálculos}& Con.Radio.Ext & Con.Radio.Int & Con.Altura\\
\hline
Cilin.H1 & 0.00034	& 0.00013 & 0.00059\\
\hline
Cilin.H2 &  0.00084	& 0.00031 & 0.0018\\
\hline
Cilin.H3 & 0.0017	& 0.00027 & 0.00012\\
\hline
Cilin.H4 & 3.00e-05	&  0.00078 & 0.00056\\
\hline
Cilin.H5 &  0.00024	&  0.0051 & 0.00061\\
\hline
Cilin.H6 &  0.0013	&  0.0021 &  0.00012\\
\hline
Cilin.H7 & 0.0016	& 0.0012 & 0.00040\\
\hline
Cilin.H8 & 0.00096	& 0.0060 & 0.0013\\
\hline
Cilin.H9 &  0.0014	& 0.024 &  0.0034\\
\hline
Cilin.H10 & 0.0028	& 0.0099 & 0.0014\\
\hline
\end{tabular}
\caption{\centering Contribución del radio interno, el radio externo y la altura al error de la funcion del volumen en cada test del cilindro hueco}
\label{Tabla19}
\end{table}
\hfill \break
Aplicando la Ec.~\eqref{ec:16}, con la finalidad de calcular el error de la función del volumen, utilizando las contribuciones del radio interno, el radio externo y la altura calculadas en cada test del cilindro hueco.

\begin{table}[H]
\centering
\begin{tabular}{|l|c|}
\hline
\backslashbox{Test}{Cálculos}& Error en la función del Volumen\\
\hline
Cilin.H1 & 0.00069\\
\hline
Cilin.H2 & 	0.0020\\
\hline
Cilin.H3 &  0.0017	\\
\hline
Cilin.H4 & 0.00096\\
\hline
Cilin.H5 &  0.0052	\\
\hline
Cilin.H6 &   0.0025	\\
\hline
Cilin.H7 &   0.0020	\\
\hline
Cilin.H8 &  0.0062	\\
\hline
Cilin.H9 &  0.025	\\
\hline
Cilin.H10 &  0.010	\\
\hline
\end{tabular}
\caption{\centering Error en la función del volumen en cada test del cilindro hueco}
\label{Tabla20}
\end{table}
\hfill \break
Aplicando la Ec.~\eqref{ec:13} y la Ec.~\eqref{ec:14} a la ecuación de la densidad Ec.~\eqref{ec:5}, con la finalidad de calcular la contribución de la masa y del volumen al error de la función de la densidad, utilizando las medias y los errores de la media de la masa y el volumen obtenidos en cada test del cilindro hueco.

\begin{table}[H]
\centering
\begin{tabular}{|l|c|c|}
\hline
\backslashbox{Test}{Cálculos}& Contribución.Masa  & Contribución.Volumen\\
\hline
Cilin.H1 & 1.2e-05	 & 7.5e-07\\
\hline
Cilin.H2 & 1.2e-05	 & 4.1e-06\\
\hline
Cilin.H3 & 3.2e-06	 &  9.8e-06\\
\hline
Cilin.H4 & 	1.1e-06 & 2.4e-06\\
\hline
Cilin.H5 & 7.9e-07	 & 3.1e-06\\
\hline
Cilin.H6 & 	2.6e-06 & 1.6e-05\\
\hline
Cilin.H7 & 3.2e-06	 &  1.2e-05\\
\hline
Cilin.H8 & 1.8-06	 & 5.8e-06\\
\hline
Cilin.H9 &  2.8e-08	 & 8.5e-07\\
\hline
Cilin.H10 & 1.0e-06	 & 1.1e-06\\
\hline
\end{tabular}
\caption{\centering Contribución de la masa y el volumen al error de la función de la densidad en cada test del cilindro hueco}
\label{Tabla21}
\end{table}
\hfill \break
Aplicando la Ec.~\eqref{ec:16}, con la finalidad de calcular el error de la función de la densidad, utilizando las contribuciones de la masa y el volumen calculadas en cada test del cilindro hueco.

\begin{table}[H]
\centering
\begin{tabular}{|l|c|}
\hline
\backslashbox{Test}{Cálculos}& Error en la función de la Densidad\\
\hline
Cilin.H1 & 	1.2e-05\\
\hline
Cilin.H2 & 	 1.2e-05\\
\hline
Cilin.H3 & 	1.0e-05\\
\hline
Cilin.H4 & 	2.7e-06\\
\hline
Cilin.H5 & 	3.2-06\\
\hline
Cilin.H6 &  1.6e-05	\\
\hline
Cilin.H7 & 	 1.3e-05\\
\hline
Cilin.H8 & 	 6.1e-06\\
\hline
Cilin.H9 & 	8.5e-07\\
\hline
Cilin.H10 & 1.5e-06\\
\hline
\end{tabular}
\caption{\centering Error en la función de la Densidad en cada test del cilindro hueco}
\label{Tabla22}
\end{table}

\subsection{MEDICIONES DEL CILINDRO}
\hfill \break
Obteniendo datos de la media de la altura, el radio y la masa del cilindro, previamente calculados en el trabajo de medidas indirectas realizada por Mateo Merino Vidal\cite{202301308Prt0230032023}, a través de la Ec.~\eqref{ec:7}.

\begin{table}[H]
\centering
\begin{tabular}{|c|c|c|c|}
\hline
Test & Altura [cm] & Radio [cm]  & Masa [gr]\\
\hline
Cilindro1 & 1.37 & 0.34 & 17.86\\
\hline
Cilindro2 & 2.40 & 0.86 & 21.53\\
\hline
Cilindro3 & 3.41 & 0.95 & 22.42\\
\hline
Cilindro4 & 4.11 & 1.01 & 24.60\\
\hline
Cilindro5 & 4.73 & 1.11 & 81.24\\
\hline
Cilindro6 & 5.67 & 1.52 & 88.25\\
\hline
Cilindro7 & 5.79 & 1.93 & 93.92\\
\hline
Cilindro8 & 9.29 & 1.99 & 99.44\\
\hline
Cilindro9 & 9.46 & 2.03 & 119.89\\
\hline
Cilindro10 & 9.97 & 2.11 & 120.95\\
\hline
\end{tabular}
\caption{\centering Valor de tendencia central de los datos obtenidos de la altura, el radio y la masa del cilindro en cada test.}
\label{tabla23}
\end{table}
\hfill \break
Obteniendo datos de la media del volumen del cilindro, previamente calculados en el trabajo de medidas indirectas realizada por Mateo Merino Vidal\cite{202301308Prt0230032023}, a través de la Ec.~\eqref{ec:2}, para hallar los volúmenes individuales en base a las medidas directas obtenidas de la medición de las dimensiones del cilindro y la Ec.~\eqref{ec:7}, para hallar la media de esos volúmenes en cada test del cilindro.

\begin{table}[H]
\centering
\begin{tabular}{|c|c|}
\hline
Test & Media del Volumen [$cm^3$]\\
\hline
Cilindro1 & 0.51\\
\hline
Cilindro2 & 5.55\\
\hline
Cilindro3 & 9.66\\
\hline
Cilindro4 & 13.28\\
\hline
Cilindro5 & 18.27\\
\hline
Cilindro6 & 41.04 \\
\hline
Cilindro7 & 67.65\\
\hline
Cilindro8 & 115.77\\
\hline
Cilindro9 & 122.13\\
\hline
Cilindro10 & 139.24 \\
\hline
\end{tabular}
\caption{ \centering Promedio del volumen de cada test}
\label{Tabla24}
\end{table}

\begin{figure}[H]
\centering
\includegraphics[scale=0.55]{Imagen10.png}
\caption{\centering Media del volumen del cilindro en cada test}
\label{Figura5}
\end{figure}
\hfill \break
Obteniendo datos de la media de la densidad del cilindro previamente calculados en el trabajo de medidas indirectas realizada por Mateo Merino Vidal\cite{202301308Prt0230032023}, a través de la Ec.~\eqref{ec:5}, para hallar las densidades individuales en base a los datos obtenidos de masa, volumen y la Ec.~\eqref{ec:7}, para hallar la media de esas densidades en cada test del cilindro.

\begin{table}[H]
\centering
\begin{tabular}{|c|c|}
\hline
Test & Media de la Densidad [$gr/cm^3$]\\
\hline
Cilindro1 & 34.99\\
\hline
Cilindro2 & 3.88\\
\hline
Cilindro3 & 2.32 \\
\hline
Cilindro4 & 1.85\\
\hline
Cilindro5 & 4.44\\
\hline
Cilindro6 & 2.15\\
\hline
Cilindro7 & 1.39\\
\hline
Cilindro8 & 0.86 \\
\hline
Cilindro9 & 0.98\\
\hline
Cilindro10 & 0.87\\
\hline
\end{tabular}
\caption{ \centering Promedio de la densidad de cada test}
\label{Tabla25}
\end{table}

\begin{figure}[H]
\centering
\includegraphics[scale=0.55]{Imagen11.png}
\caption{\centering Media de la densidad del cilindro en cada test}
\label{Figura6}
\end{figure}
\hfill \break
Aplicando la ecuación ~\eqref{ec:8}, utilizando la media de la altura, el radio y la masa del cilindro obtenidos anteriormente, con la finalidad de encontrar sus desviaciones individuales en cada uno de los tests.
\hfill \break
\hfill \break
Aplicando la Ec. ~\eqref{ec:9} a las desviaciones individuales de la altura, el radio y la masa obtenidos en cada test del cilindro, con la finalidad de determinar la desviación típica de cada uno en los diversos tests.
\hfill \break
\hfill \break
Aplicando la Ec.~\eqref{ec:11} a las desviaciones típicas de la altura, el radio y la masa obtenidos anteriormente en cada test, con la finalidad de calcular el error de la media en cada uno de los tests del cilindro.

\hfill \break
\textbf{Altura} 

\begin{table}[H]
\centering
\begin{tabular}{|l|c|c|c|}
\hline
\backslashbox{Test}{Cálculos} & Media [cm] & Des.Estándar & Error.Media\\
\hline
Cilindro1 &  1.38 & 3.7e-05	&  1.9e-06	\\
\hline
Cilindro2 &  2.40 &  0.0019	& 9.6e-05	\\
\hline
Cilindro3 & 3.41  &  4.2e-07	& 2.1e-08	\\
\hline
Cilindro4 & 4.12  & 0.00094	& 4.7e-05	\\
\hline
Cilindro5 & 4.74  & 0.0017	& 8.6e-05	\\
\hline
Cilindro6 &  5.67  & 0.00094	& 	4.7e-05\\
\hline
Cilindro7 & 5.79  & 0.00016	&  7.8e-06	\\
\hline
Cilindro8 &   9.29 & 0.0022	& 0.00011	\\
\hline
Cilindro9 &   9.46 &  7.3e-05	& 3.6e-06	\\
\hline
Cilindro10 & 9.97  & 0.00072	& 3.6e-05	\\
\hline
\end{tabular}
\caption{\centering Punto de tendencia central, desviaciones y error de la media de la altura del cilindro en cada test}
\label{Tabla26}
\end{table}
\hfill \break
\textbf{Radio} 

\begin{table}[H]
\centering
\begin{tabular}{|l|c|c|c|}
\hline
\backslashbox{Test}{Cálculos} & Media [cm] & Des.Estándar & Error.Media\\
\hline
Cilindro1 & 0.34  & 0.00014	& 6.9e-06	\\
\hline
Cilindro2 &  0.86 & 4.6e-06	& 	2.3e-07\\
\hline
Cilindro3 &  0.95 & 0.00023	& 1.2e-05	\\
\hline
Cilindro4 &  1.01 & 0.00011	&  5.6e-06	\\
\hline
Cilindro5 &  1.11 &  8.9e-05	& 	4.4e-06\\
\hline
Cilindro6 &  1.52 &  0.00013	& 	6.4e-06\\
\hline
Cilindro7 &   1.93 &  0.00015	& 7.6e-06	\\
\hline
Cilindro8 &  2.00 & 0.00021	&  1.1e-05	\\
\hline
Cilindro9 & 2.03  & 8.3e-05	& 4.2e-06	\\
\hline
Cilindro10 & 2.11  &  2.9e-05	& 	1.4e-06\\
\hline
\end{tabular}
\caption{\centering Punto de tendencia central, desviaciones y error de la media del radio del cilindro en cada test}
\label{Tabla27}
\end{table}
\hfill \break
\textbf{Masa} 

\begin{table}[H]
\centering
\begin{tabular}{|l|c|c|c|}
\hline
\backslashbox{Test}{Cálculos} & Media [gr] & Des.Estándar & Error.Media\\
\hline
Cilindro1 & 17.86  & 0.0069	& 0.00035	\\
\hline
Cilindro2 &  21.53  & 0.012	& 0.00062	\\
\hline
Cilindro3 & 22.42  & 0.0023	&  0.00012	\\
\hline
Cilindro4 & 24.60  & 0.00013	&  6.6e-06	\\
\hline
Cilindro5 &  81.24  & 0.030	& 	 0.0015\\
\hline
Cilindro6 & 88.25  & 0.0026	& 	 0.00013\\
\hline
Cilindro7 &  93.92 & 0.026	& 	0.0013\\
\hline
Cilindro8 & 99.44  & 0.012	& 0.00058	\\
\hline
Cilindro9 &  119.89 & 0.016	& 0.00082	\\
\hline
Cilindro10 & 120.95  & 0.0073	& 0.00037	\\
\hline
\end{tabular}
\caption{\centering Punto de tendencia central, desviaciones y error de la media de la masa del cilindro en cada test}
\label{Tabla28}
\end{table}
\hfill \break
Obteniendo datos del error de la media del volumen, previamente calculados en el trabajo de medidas indirectas realizada por Mateo Merino Vidal\cite{202301308Prt0230032023}, a través de la Ec.~\eqref{ec:11}, utilizando la desviación típica calculada mediante la Ec.~\eqref{ec:9} en base a las desviaciones individuales de los volúmenes, determinadas mediante la Ec.~\eqref{ec:8}.

\hfill \break
\textbf{Volumen} 

\begin{table}[H]
\centering
\begin{tabular}{|l|c|}
\hline
\backslashbox{Test}{Cálculos} & Error.Media\\
\hline
Cilindro1 & 2.0e-05	\\
\hline
Cilindro2 & 0.00022	\\
\hline
Cilindro3 & 0.00024	\\
\hline
Cilindro4 &  4.6e-06	\\
\hline
Cilindro5 & 	0.00048\\
\hline
Cilindro6 &  0.00069	\\
\hline
Cilindro7 &   0.00044	\\
\hline
Cilindro8 &   0.00014	\\
\hline
Cilindro9 &  0.00055	\\
\hline
Cilindro10 &  	0.00031\\
\hline
\end{tabular}
\caption{\centering Punto de tendencia central, desviaciones y error de la media en el volumen del cilindro en cada test}
\label{Tabla29}
\end{table}
\hfill \break
Aplicando la Ec.~\eqref{ec:13},Ec.~\eqref{ec:14},Ec.~\eqref{ec:15} a la ecuación del volumen Ec.~\eqref{ec:2}, con la finalidad de calcular la contribución del radio y la altura al error de la función del volumen, utilizando las medias y los errores de la media de los mismos obtenidos en cada test del cilindro.

\begin{table}[H]
\centering
\begin{tabular}{|l|c|c|}
\hline
\backslashbox{Test}{Cálculos} & Contribución.Radio & Contribución.Altura\\
\hline
Cilindro1 & 2.0e-05	&  6.9e-07\\
\hline
Cilindro2 & 8.9e-05	& 0.00022 \\
\hline
Cilindro3 & 0.00014	& 5.9e-08 \\
\hline
Cilindro4 & 0.00018	& 0.00015 \\
\hline
Cilindro5 & 0.00023	&  0.00033\\
\hline
Cilindro6 & 0.00037	& 0.00034 \\
\hline
Cilindro7 & 0.00048	&  9.1e-05\\
\hline
Cilindro8 & 0.00080	& 0.0014 \\
\hline
Cilindro9 & 0.00083	&  4.7e-05\\
\hline
Cilindro10 & 0.00091	&  0.00050\\
\hline
\end{tabular}
\caption{\centering Contribución del radio y la altura al error de la función del volumen en cada test del cilindro}
\label{Tabla30}
\end{table}
\hfill \break
Aplicando la Ec.~\eqref{ec:16}, con la finalidad de calcular el error de la función del volumen, utilizando las contribuciones del radio y la altura calculadas en cada test del cilindro.

\begin{table}[H]
\centering
\begin{tabular}{|l|c|}
\hline
\backslashbox{Test}{Cálculos}& Error en la función del Volumen\\
\hline
Cilindro1 & 2.0e-05\\
\hline
Cilindro2 & 0.00024\\
\hline
Cilindro3 & 0.00014\\
\hline
Cilindro4 & 0.00024\\
\hline
Cilindro5 & 0.00040\\
\hline
Cilindro6 & 0.00050\\
\hline
Cilindro7 & 0.00049\\
\hline
Cilindro8 & 0.0016\\
\hline
Cilindro9 & 0.00083\\
\hline
Cilindro10 & 0.0010\\
\hline
\end{tabular}
\caption{\centering Error en la función del volumen en cada test del cilindro}
\label{Tabla31}
\end{table}
\hfill \break
Aplicando la Ec.~\eqref{ec:13} y la Ec.~\eqref{ec:14} a la ecuación de la densidad Ec.~\eqref{ec:5}, con la finalidad de calcular la contribución de la masa y del volumen al error de la función de la densidad, utilizando las medias y los errores de la media de la masa y el volumen obtenidos en cada test del cilindro.

\begin{table}[H]
\centering
\begin{tabular}{|l|c|c|}
\hline
\backslashbox{Test}{Cálculos}& Contribución.Masa  & Contribución.Volumen\\
\hline
Cilindro1 & 0.00068	 & 0.0013\\
\hline
Cilindro2 & 0.00011	 & 0.00015\\
\hline
Cilindro3 & 1.2e-05	 & 5.7e-05\\
\hline
Cilindro4 &  5.0e-07	 &  6.5e-07\\
\hline
Cilindro5 & 8.1e-05	 & 0.00012\\
\hline
Cilindro6 & 3.2e-06	 & 3.6e-05\\
\hline
Cilindro7 &  1.9e-05 &  9.1e-06\\
\hline
Cilindro8 & 5.0e-06	 & 1.0e-06\\
\hline
Cilindro9 & 6.7e-06	 & 4.4e-06\\
\hline
Cilindro10 & 2.6e-06	 & 1.9e-06\\
\hline
\end{tabular}
\caption{\centering Contribución de la masa y el volumen al error de la función de la densidad en cada test del cilindro}
\label{Tabla32}
\end{table}
\hfill \break
Aplicando la Ec.~\eqref{ec:16}, con la finalidad de calcular el error de la función de la densidad, utilizando las contribuciones de la masa y el volumen calculadas en cada test del cilindro.

\begin{table}[H]
\centering
\begin{tabular}{|l|c|}
\hline
\backslashbox{Test}{Cálculos}& Error en la función de la Densidad\\
\hline
Cilindro1 & 	 0.0015\\
\hline
Cilindro2 & 	 0.00019\\
\hline
Cilindro3 & 	 5.8e-05\\
\hline
Cilindro4 & 	8.2e-07\\
\hline
Cilindro5 & 	0.00014\\
\hline
Cilindro6 & 	3.6e-05\\
\hline
Cilindro7 & 	 2.1e-05\\
\hline
Cilindro8 & 	5.1e-06\\
\hline
Cilindro9 & 	 8.0e-06\\
\hline
Cilindro10 & 	3.3e-06\\
\hline
\end{tabular}
\caption{\centering Error en la función de la Densidad en cada test del cilindro}
\label{Tabla33}
\end{table}

\subsection{MEDICIONES DE LA MONEDA}
\hfill \break
Obteniendo datos de la media del diámetro, el espesor y la masa de la moneda, previamente calculados en el trabajo de medidas indirectas realizada por Mateo Merino Vidal\cite{202301308Prt0230032023}, a través de la Ec.~\eqref{ec:7}.

\begin{table}[H]
\centering
\begin{tabular}{|c|c|c|c|}
\hline
Test & Diámetro [cm] & Espesor [cm]  & Masa [gr]\\
\hline
Moneda1 & 1.89 & 0.28 & 0.32\\
\hline
Moneda2 & 2.18 & 0.41 & 0.34\\
\hline
Moneda3 & 2.25 & 0.41 & 0.36\\
\hline
Moneda4 & 2.97 & 0.41 & 0.38\\
\hline
Moneda5 & 3.21 & 0.48 & 0.50\\
\hline
Moneda6 & 4.17 & 0.52 & 0.59\\
\hline
Moneda7 & 4.26 & 0.58 & 0.63\\
\hline
Moneda8 & 5.02 & 0.61 & 0.83\\
\hline
Moneda9 & 5.35 & 0.62 & 0.94\\
\hline
Moneda10 & 5.46 & 0.66 & 1.18\\
\hline
\end{tabular}
\caption{\centering Valor de tendencia central de los datos obtenidos del diámetro, el espesor y la masa de la moneda en cada test.}
\label{tabla34}
\end{table}
\hfill \break
Obteniendo datos de la media del volumen de la moneda, previamente calculados en el trabajo de medidas indirectas realizada por Mateo Merino Vidal\cite{202301308Prt0230032023}, a través de la Ec.~\eqref{ec:4}, para hallar los volúmenes individuales en base a las medidas directas obtenidas de la medición de las dimensiones de la moneda y la Ec.~\eqref{ec:7}, para hallar la media de esos volúmenes en cada test de la moneda.

\begin{table}[H]
\centering
\begin{tabular}{|c|c|}
\hline
Test & Media del Volumen [$cm^3$]\\
\hline
Moneda1 & 0.77 \\
\hline
Moneda2 & 1.52\\
\hline
Moneda3 & 1.61\\
\hline
Moneda4 & 2.81\\
\hline
Moneda5 & 3.86 \\
\hline
Moneda6 & 7.12\\
\hline
Moneda7 & 8.21\\
\hline
Moneda8 & 12.09 \\
\hline
Moneda9 & 14.01\\
\hline
Moneda10 & 15.52\\
\hline
\end{tabular}
\caption{ \centering Promedio del volumen de cada test}
\label{Tabla35}
\end{table}

\begin{figure}[H]
\centering
\includegraphics[scale=0.55]{Imagen14.png}
\caption{\centering Media del volumen de la moneda en cada test}
\label{Figura7}
\end{figure}
\hfill \break
Obteniendo datos de la media de la densidad de la moneda previamente calculados en el trabajo de medidas indirectas realizada por Mateo Merino Vidal\cite{202301308Prt0230032023}, a través de la Ec.~\eqref{ec:5}, para hallar las densidades individuales en base a los datos obtenidos de masa, volumen y la Ec.~\eqref{ec:7}, para hallar la media de esas densidades en cada test de la moneda.

\begin{table}[H]
\centering
\begin{tabular}{|c|c|}
\hline
Test & Media de la Densidad[$gr/cm^3$]\\
\hline
Moneda1 & 0.41\\
\hline
Moneda2 & 0.22\\
\hline
Moneda3 & 0.22\\
\hline
Moneda4 & 0.14\\
\hline
Moneda5 & 0.13\\
\hline
Moneda6 & 0.082\\
\hline
Moneda7 & 0.077\\
\hline
Moneda8 & 0.068 \\
\hline
Moneda9 & 0.067\\
\hline
Moneda10 & 0.076\\
\hline
\end{tabular}
\caption{ \centering Promedio de la densidad de cada test}
\label{Tabla36}
\end{table}

\begin{figure}[H]
\centering
\includegraphics[scale=0.55]{Imagen15.png}
\caption{\centering Media de la densidad de la moneda en cada test}
\label{Figura8}
\end{figure}
\hfill \break
Aplicando la ecuación ~\eqref{ec:8}, utilizando la media del diámetro, el espesor y la masa de la moneda obtenidos anteriormente, con la finalidad de encontrar sus desviaciones individuales en cada uno de los tests.
\hfill \break
\hfill \break
Aplicando la Ec. ~\eqref{ec:9} a las desviaciones individuales del diámetro, el espesor y la masa obtenidos en cada test de la moneda, con la finalidad de determinar la desviación típica de cada uno en los diversos tests.
\hfill \break
\hfill \break
Aplicando la Ec.~\eqref{ec:11} a las desviaciones típicas del diámetro, el espesor y la masa obtenidos anteriormente en cada test, con la finalidad de calcular el error de la media en cada uno de los tests de la moneda.

\hfill \break
\textbf{Diámetro} 

\begin{table}[H]
\centering
\begin{tabular}{|l|c|c|c|}
\hline
\backslashbox{Test}{Cálculos} & Media [cm] & Des.Estándar & Error.Media\\
\hline
Moneda1 & 1.88  & 0.00035	& 1.7e-05	\\
\hline
Moneda2 &  2.18  & 0.00045	& 2.2e-05	\\
\hline
Moneda3 & 2.25  &  0.00034	& 	1.7e-05\\
\hline
Moneda4 &   2.97 & 5.04	& 2.5e-06	\\
\hline
Moneda5 &  3.21 & 0.00034	& 1.7e-05	\\
\hline
Moneda6 &   4.17 & 0.00047	& 	 2.4e-05\\
\hline
Moneda7 & 4.26  & 0.00074	& 	3.7e-05\\
\hline
Moneda8 &  5.02 & 0.00038	& 	 1.9e-05\\
\hline
Moneda9 & 5.35  & 7.8e-05	& 3.9e-06	\\
\hline
Moneda10 & 5.46  &  0.00023	& 1.2e-05	\\
\hline
\end{tabular}
\caption{\centering Punto de tendencia central, desviaciones y error de la media del diámetro de la moneda en cada test}
\label{Tabla37}
\end{table}
\hfill \break
\textbf{Espesor} 

\begin{table}[H]
\centering
\begin{tabular}{|l|c|c|c|}
\hline
\backslashbox{Test}{Cálculos} & Media [cm] & Des.Estándar & Error.Media\\
\hline
Moneda1 & 0.28  &  1.6e-05	& 7.9e-07	\\
\hline
Moneda2 &  0.40  &  4.1e-05	& 	 2.1e-06\\
\hline
Moneda3 &  0.40 & 2.5e-05	&  1.2e-06	\\
\hline
Moneda4 &   0.41 & 7.7e-05	& 3.8e-06	\\
\hline
Moneda5 &  0.48 & 2.9e-05	& 	 1.4e-06\\
\hline
Moneda6 &  0.52  & 5.8e-05	& 2.9e-06	\\
\hline
Moneda7 &  0.58 & 2.8e-05	& 	1.4e-06\\
\hline
Moneda8 &  0.61 &  2.3e-05	& 1.2e-06	\\
\hline
Moneda9 & 0.62  & 2.6e-05	&  1.3e-06	\\
\hline
Moneda10 &  0.66  &  1.0e-05	& 	5.0e-07\\
\hline
\end{tabular}
\caption{\centering Punto de tendencia central, desviaciones y error de la media del espesor de la moneda en cada test}
\label{Tabla38}
\end{table}
\hfill \break
\textbf{Masa} 

\begin{table}[H]
\centering
\begin{tabular}{|l|c|c|c|}
\hline
\backslashbox{Test}{Cálculos} & Media [gr] & Des.Estándar & Error.Media\\
\hline
Moneda1 & 0.32  &  3.9e-05	& 	2.00e-06\\
\hline
Moneda2 &   0.34 &  2.7e-05	& 1.3e-06	\\
\hline
Moneda3 & 0.36  &  1.7e-05	& 	 8.6e-07\\
\hline
Moneda4 &  0.38  & 0.00012	& 6.0e-06	\\
\hline
Moneda5 & 0.50  & 0.00012	&  5.9e-06	\\
\hline
Moneda6 &  0.59 & 0.00016	& 7.8e-06	\\
\hline
Moneda7 &  0.63  & 4.5e-05	&  2.2e-06	\\
\hline
Moneda8 & 0.83  &  9.3e-05	& 	4.6e-06\\
\hline
Moneda9 &  0.94 & 8.8e-05	& 4.4e-06 	\\
\hline
Moneda10 & 1.18  & 7.6e-05	& 	3.8e-06\\
\hline
\end{tabular}
\caption{\centering Punto de tendencia central, desviaciones y error de la media de la masa de la moneda en cada test}
\label{Tabla39}
\end{table}
\hfill \break
Obteniendo datos del error de la media del volumen, previamente calculados en el trabajo de medidas indirectas realizada por Mateo Merino Vidal\cite{202301308Prt0230032023}, a través de la Ec.~\eqref{ec:11}, utilizando la desviación típica calculada mediante la Ec.~\eqref{ec:9} en base a las desviaciones individuales de los volúmenes, determinadas mediante la Ec.~\eqref{ec:8}.

\hfill \break
\textbf{Volumen} 

\begin{table}[H]
\centering
\begin{tabular}{|l|c|}
\hline
\backslashbox{Test}{Cálculos} & Error de la Media\\
\hline
Moneda1 & 1.6e-05\\
\hline
Moneda2 & 3.9e-05	\\
\hline
Moneda3 & 2.0e-05	\\
\hline
Moneda4 &  3.1e-05	\\
\hline
Moneda5 &   5.2e-05	\\
\hline
Moneda6 &  0.00012	\\
\hline
Moneda7 &  	 0.00012\\
\hline
Moneda8 &  	6.8e-05\\
\hline
Moneda9 &  5.0e-05	\\
\hline
Moneda10 &  7.8e-05\\
\hline
\end{tabular}
\caption{\centering Punto de tendencia central, desviaciones y error de la media en el volumen de la moneda en cada test}
\label{Tabla40}
\end{table}
\hfill \break
Aplicando la Ec.~\eqref{ec:13},Ec.~\eqref{ec:14},Ec.~\eqref{ec:15} a la ecuación del volumen Ec.~\eqref{ec:4}, con la finalidad de calcular la contribución del diámetro y el espesor al error de la función del volumen, utilizando las medias y los errores de la media de los mismos obtenidos en cada test de la moneda.

\begin{table}[H]
\centering
\begin{tabular}{|l|c|c|}
\hline
\backslashbox{Test}{Cálculos} & Contribución.Diámetro & Contribución.Espesor\\
\hline
Moneda1 & 1.4e-05	&  2.2e-06\\
\hline
Moneda2 & 3.1e-05	&  7.8e-06\\
\hline
Moneda3 &  2.5e-05	&   4.9e-06\\
\hline
Moneda4 & 4.8e-06	& 2.7e-05 \\
\hline
Moneda5 & 4.1e-05	& 1.2e-05 \\
\hline
Moneda6 & 8.1e-05	&  4.0e-05\\
\hline
Moneda7 & 0.00014	&   1.98\\
\hline
Moneda8 & 9.1e-05	& 2.3e-05 \\
\hline
Moneda9 &  2.0e-05	&  2.9e-05 \\
\hline
Moneda10 & 6.6e-05	& 1.2e-05 \\
\hline
\end{tabular}
\caption{\centering Contribución del diámetro y el espesor al error de la función del volumen en cada test de la moneda}
\label{Tabla41}
\end{table}
\hfill \break
Aplicando la Ec.~\eqref{ec:16}, con la finalidad de calcular el error de la función del volumen, utilizando las contribuciones del diámetro y el espesor calculadas en cada test de la moneda.

\begin{table}[H]
\centering
\begin{tabular}{|l|c|}
\hline
\backslashbox{Test}{Cálculos}& Error en la función del Volumen\\
\hline
Moneda1 &  1.4e-05\\
\hline
Moneda2 & 3.2e-05\\
\hline
Moneda3 &  2.5e-05\\
\hline
Moneda4 & 2.7e-05\\
\hline
Moneda5 & 4.2-05\\
\hline
Moneda6 &  9.0e-05\\
\hline
Moneda7 & 0.00014\\
\hline
Moneda8 & 9.4e-05\\
\hline
Moneda9 & 3.6e-05\\
\hline
Moneda10 & 6.7e-05\\
\hline
\end{tabular}
\caption{\centering Error en la función del volumen en cada test de la moneda}
\label{Tabla42}
\end{table}
\hfill \break
Aplicando la Ec.~\eqref{ec:13} y la Ec.~\eqref{ec:14} a la ecuación de la densidad Ec.~\eqref{ec:5}, con la finalidad de calcular la contribución de la masa y del volumen al error de la función de la densidad, utilizando las medias y los errores de la media de la masa y el volumen obtenidos en cada test de la moneda.

\begin{table}[H]
\centering
\begin{tabular}{|l|c|c|}
\hline
\backslashbox{Test}{Cálculos}& Contribución.Masa  & Contribución.Volumen\\
\hline
Moneda1  & 2.5e-06	 & 8.8e-06\\
\hline
Moneda2  & 	8.8e-07 & 5.8e-06\\
\hline
Moneda3  & 5.3e-07	 &  2.8e-06\\
\hline
Moneda4  & 2.1e-06	 & 1.5e-06\\
\hline
Moneda5  & 	1.5e-06 & 1.8e-06\\
\hline
Moneda6  & 1.1e-06	 & 1.4e-06\\
\hline
Moneda7  & 2.7e-07	 & 1.2e-06\\
\hline
Moneda8  & 3.8e-07	 & 3.8e-07\\
\hline
Moneda9  & 3.2e-07	 &  2.4e-07\\
\hline
Moneda10  & 2.4e-07	 & 3.8e-07\\
\hline
\end{tabular}
\caption{\centering Contribución de la masa y el volumen al error de la función de la densidad en cada test de la moneda}
\label{Tabla43}
\end{table}
\hfill \break
Aplicando la Ec.~\eqref{ec:16}, con la finalidad de calcular el error de la función de la densidad, utilizando las contribuciones de la masa y el volumen calculadas en cada test de la moneda.

\begin{table}[H]
\centering
\begin{tabular}{|l|c|}
\hline
\backslashbox{Test}{Cálculos}& Error en la función de la Densidad\\
\hline
Moneda1 & 	9.1e-06\\
\hline
Moneda2 & 	5.8e-06\\
\hline
Moneda3 & 	 2.8e-06\\
\hline
Moneda4 & 	 2.6e-06\\
\hline
Moneda5 & 	2.3e-06\\
\hline
Moneda6 & 	1.8e-06\\
\hline
Moneda7 & 	1.2e-06\\
\hline
Moneda8 & 	 5.4e-07\\
\hline
Moneda9 & 	3.9e-07\\
\hline
Moneda10 & 	4.5e-07\\
\hline
\end{tabular}
\caption{\centering Error en la función de la Densidad en cada test de la moneda}
\label{Tabla44}
\end{table}

\section{Resultados}
\label{Sec:res}
\subsection{ESFERA}
\hfill \break
\textbf{Volumen} 
\hfill \break
Se puede observar en la Fig.\ref{Figura1} que en el caso del volumen, los tests (1,2,3,4,5,6) de los datos obtenidos en los tests de la esfera en la Tabla.\ref{Tabla8} poseen un mínimo error de la función del volumen menor a $0.0020$, además de que al ser directamente proporcional a las medidas directas tomadas en la mensuración de las dimensiones de la esfera, evidencia que las medidas directas fueron tomadas de manera muy precisa, con un instrumento correctamente calibrado.
\hfill \break
\hfill \break
Sin embargo, los tests (7,8,9,10) de los datos obtenidos en los tests de la esfera en la Tabla.\ref{Tabla8} poseen un error de la función del volumen mayor a $0.0030$.
\hfill \break
\hfill \break
Analizando los tests en general, en base a la Fig.\ref{Figura1}, se evidencia que los datos están muy dispersos y que su distribucion no es simétrica como en la campana de Gauss, sino sesgada a uno de los extremos.
\hfill \break
\hfill \break
Asimismo se observa que la función que estamos observando no es mas que una función exponencial, de forma creciente.
\hfill \break
\hfill \break
\textbf{Densidad} 
\hfill \break
Se puede observar en la Fig.\ref{Figura2} que en el caso de la densidad, los tests (2,3,4,5,6,7,8,9,10) de los datos obtenidos en los tests de la esfera en la Tabla.\ref{Tabla10} poseen un mínimo error de la función de la densidad menor a $0.000099$, además de que al ser directamente proporcional a las medidas directas tomadas en la mensuración de las dimensiones de la esfera, evidencia que las medidas directas fueron tomadas de manera muy precisa, con un instrumento correctamente calibrado.
\hfill \break
\hfill \break
Sin embargo, el test2 de los datos obtenidos en los tests de la esfera en la Tabla.\ref{Tabla10} posee un error de la función de la densidad mayor a $0.00010$.
\hfill \break
\hfill \break
Analizando los tests en general, en base a la Fig.\ref{Figura2}, se evidencia que los datos están concentrados en un solo lugar y que su distribucion es simétrica como en la campana de Gauss.

\subsection{CILINDRO HUECO}
\hfill \break
\textbf{Volumen} 
\hfill \break
Se puede observar en la Fig.\ref{Figura3} que en el caso del volumen, los tests (1,2,3,4,5,6,7) de los datos obtenidos en los tests del cilindro hueco en la Tabla.\ref{Tabla20} poseen un mínimo error de la función del volumen menor a $0.0030$, además de que al ser directamente proporcional a las medidas directas tomadas en la mensuración de las dimensiones del cilindro hueco, evidencia que las medidas directas fueron tomadas de manera muy precisa, con un instrumento correctamente calibrado.
\hfill \break
\hfill \break
Sin embargo, los tests (8,9,10) de los datos obtenidos en los tests del cilindro hueco en la Tabla.\ref{Tabla20} poseen un error de la función del volumen mayor a $0.0050$.
\hfill \break
\hfill \break
Analizando los tests en general, en base a la Fig.\ref{Figura3}, se evidencia que los datos están muy dispersos y que su distribucion no es simétrica como en la campana de Gauss, sino sesgada a uno de los extremos.
\hfill \break
\hfill \break
Asimismo se observa que la función que estamos observando no es mas que una función cubica.
\hfill \break
\hfill \break
\textbf{Densidad} 
\hfill \break
Se puede observar en la Fig.\ref{Figura4} que en el caso de la densidad, los tests (4,5,8,9,10) de los datos obtenidos en los tests del cilindro hueco en la Tabla.\ref{Tabla22} poseen un mínimo error de la función de la densidad menor a $0.0000080$, además de que al ser directamente proporcional a las medidas directas tomadas en la mensuración de las dimensiones del cilindro hueco, evidencia que las medidas directas fueron tomadas de manera muy precisa, con un instrumento correctamente calibrado.
\hfill \break
\hfill \break
Sin embargo, los tests (1,2,3,6,7) de los datos obtenidos en los tests del cilindro hueco en la Tabla.\ref{Tabla22} poseen un error de la función de la densidad mayor a $0.0000095$.
\hfill \break
\hfill \break
Analizando los tests en general, en base a la Fig.\ref{Figura4}, se evidencia que los datos están dispersos y que su distribucion no es simétrica como en la campana de Gauss, sino sesgada a uno de los extremos.

\subsection{CILINDRO}
\hfill \break
\textbf{Volumen} 
\hfill \break
Se puede observar en la Fig.\ref{Figura5} que en el caso del volumen, los tests (1,2,3,4) de los datos obtenidos en los tests del cilindro en la Tabla.\ref{Tabla31} poseen un mínimo error de la función del volumen menor a $0.00030$, además de que al ser directamente proporcional a las medidas directas tomadas en la mensuración de las dimensiones del cilindro, evidencia que las medidas directas fueron tomadas de manera muy precisa, con un instrumento correctamente calibrado.
\hfill \break
\hfill \break
Sin embargo, los tests (5,6,7,8,9,10) de los datos obtenidos en los tests del cilindro en la Tabla.\ref{Tabla31} poseen un error de la función del volumen mayor a $0.00037$.
\hfill \break
\hfill \break
Analizando los tests en general, en base a la Fig.\ref{Figura5}, se evidencia que los datos están extremadamente dispersos y que su distribucion no es simétrica como en la campana de Gauss, sino sesgada a uno de los extremos.
\hfill \break
\hfill \break
Asimismo se observa que la función que estamos observando no es mas que una función exponencial, de forma creciente.
\hfill \break
\hfill \break
\textbf{Densidad} 
\hfill \break
Se puede observar en la Fig.\ref{Figura6} que en el caso de la densidad, los tests (2,3,4,5,6,7,8,9,10) de los datos obtenidos en los tests del cilindro en la Tabla.\ref{Tabla33} poseen un mínimo error de la función de la densidad menor a $0.00090$, además de que al ser directamente proporcional a las medidas directas tomadas en la mensuración de las dimensiones del cilindro, evidencia que las medidas directas fueron tomadas de manera muy precisa, con un instrumento correctamente calibrado.
\hfill \break
\hfill \break
Sin embargo, el test1 de los datos obtenidos en los tests del cilindro en la Tabla.\ref{Tabla33} posee un error de la función de la densidad mayor a $0.0010$.
\hfill \break
\hfill \break
Analizando los tests en general, en base a la Fig.\ref{Figura6}, se evidencia que los datos están concentrados en un solo lugar y que su distribucion es simétrica como en la campana de Gauss.

\subsection{MONEDA}
\hfill \break
\textbf{Volumen} 
\hfill \break
Se puede observar en la Fig.\ref{Figura7} que en el caso del volumen, los tests (1,2,3,4) de los datos obtenidos en los tests de la moneda en la Tabla.\ref{Tabla42} poseen un mínimo error de la función del volumen menor a $0,000033$, además de que al ser directamente proporcional a las medidas directas tomadas en la mensuración de las dimensiones de la moneda, evidencia que las medidas directas fueron tomadas de manera muy precisa, con un instrumento correctamente calibrado.
\hfill \break
\hfill \break
Sin embargo, los tests (5,6,7,8,9,10) de los datos obtenidos en los tests de la moneda en la Tabla.\ref{Tabla42} poseen un error de la función del volumen mayor a $0,000034$.
\hfill \break
\hfill \break
Analizando los tests en general, en base a la Fig.\ref{Figura7}, se evidencia que los datos están muy dispersos y que su distribucion no es simétrica como en la campana de Gauss, sino sesgada a uno de los extremos.
\hfill \break
\hfill \break
Asimismo se observa que la función que estamos observando no es mas que una función exponencial, de forma creciente.
\hfill \break
\hfill \break
\textbf{Densidad} 
\hfill \break
Se puede observar en la Fig.\ref{Figura8} que en el caso de la densidad, los tests (7,8,9,10) de los datos obtenidos en los tests de la moneda en la Tabla.\ref{Tabla44} poseen un mínimo error de la función de la densidad menor a $0.0000015$, además de que al ser directamente proporcional a las medidas directas tomadas en la mensuración de las dimensiones de la moneda, evidencia que las medidas directas fueron tomadas de manera muy precisa, con un instrumento correctamente calibrado.
\hfill \break
\hfill \break
Sin embargo, los tests (1,2,3,4,5,6) de los datos obtenidos en los tests de la moneda en la Tabla.\ref{Tabla44} poseen un error de la función de la densidad mayor a $0.0000016$.
\hfill \break
\hfill \break
Analizando los tests en general, en base a la Fig.\ref{Figura8}, se evidencia que los datos están extremadamente dispersos y que su distribucion no es simétrica como en la campana de Gauss, sino sesgada a uno de los extremos.
\hfill \break
Asimismo se observa que la función que estamos observando no es mas que una función exponencial, de forma decreciente.

\section{Discusión}
\label{Sec:Disc}
\hfill \break
Se observa que al analizar estadísticamente las series de medidas de cada test tomado a las dimensiones de las diversas figuras tridimensionales, estás permiten observar una clara diferencia entre la distribución de datos respecto al punto de tendencia central de cada serie, ya que en algunas series la distribución de los datos resulta ser asimétrica, indicando que la mayor concentración de datos tiende a uno de los extremos de la gráfica dándole a esta una figura similar al de una curva sesgada, pero en otros casos la misma llega a ser aproximadamente simétrica, teniendo una figura similar al de la campana de Gauss, indicando que la concentración de datos está mayormente por la media.
\hfill \break
\hfill \break
También se demuestra que los errores de las medidas indirectas dependen de la precisión con la que se tomaron las medidas directas de las dimensiones de las figuras tridimensionales.
\hfill \break
\hfill \break
En este caso el error en la función de las medidas indirectas del volumen y la densidad fue mínima, indicando que las medidas directas tomadas de la mensuración de las dimensiones de las figuras tridimensionales fueron tomadas con mucha precisión y un instrumento bien calibrado.
\hfill \break
\hfill \break
Como en el caso del test3 de la esfera en la Tabla.\ref{Tabla8} con un mínimo error de la función del volumen de $0.00010$

\section{Conclusiones}
\label{Sec:Concl}
\hfill \break
Se concluyo que:
\begin{enumerate}
\item Los datos obtenidos en cada serie de medidas directas de las mediciones realizadas a las dimensiones de las figuras tridimensionales como la esfera, el cilindro hueco, el cilindro y la moneda son distintos entre ellos, evidenciando una gran diferencia en su dispersión alrededor del punto de tendencia central, a la hora de analizar los datos. 

\item El error cometido en las medidas directas tomadas de las dimensiones de las figuras tridimensionales, también afectara al volumen y la densidad de cada figura, debido a que estas son medidas indirectas y a la hora de utilizar cálculos y ecuaciones para obtenerlos, dependen de las medidas directas anteriormente tomadas de las dimensiones de las figuras, es decir que son proporcionales.
Debido a que se tomo con mucha precisión las medidas directas, los errores de la función del volumen y la densidad en cada test de las figuras fue mínimo.\\
Como se observa:

\hfill \break
\hfill \break
-Esfera: En su volumen en el test3 de la Tabla.\ref{Tabla8}, con un error de la función del $0.00010$.
\hfill \break
\hfill \break
-Cilindro Hueco: En su densidad en el test4 de la Tabla.\ref{Tabla22}, con un error de la función del $2.7e-06$.
\hfill \break
\hfill \break
-Cilindro: En su volumen en el test1 de la Tabla.\ref{Tabla31}, con un error de la función del $2.0e-05$.
\hfill \break
\hfill \break
-Moneda: En su densidad en el test4 de la Tabla.\ref{Tabla44}, con un error de la función del $2.6e-06$.
\end{enumerate}

\section{Bibliografía}
\printbibliography
\hfill \break
\hfill \break
\hfill \break
\hfill \break
\hfill \break
\hfill \break
\hfill \break
\hfill \break
\hfill \break
\hfill \break
\hfill \break
\hfill \break
\hfill \break
\hfill \break
\hfill \break
\hfill \break
\hfill \break
\hfill \break
\hfill \break
\hfill \break
\hfill \break
\hfill \break
\hfill \break
\hfill \break
\hfill \break
\hfill \break
\hfill \break
\hfill \break
\hfill \break
\hfill \break
\hfill \break
\hfill \break
\hfill \break
\hfill \break
\hfill \break
\hfill \break
\hfill \break
\hfill \break

\end{document}